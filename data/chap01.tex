% !TeX root = ../main.tex

\chapter{引言}
\label{cha:intro}

互联网正逐步变为数字化社会,越来越多的商业和社会活动从线下转移到线 上。例如电子商务,电子办公等。随着智能设备以及相关应用的范围扩大,从个人电脑增加到手机、智能手环、智能电视、智能音箱等多种联网设备。大量个人数据被收 集并且上传到互联网。这些数据分布在网络中,被许多大公司掌握。另一方面,为 了丰富用户的应用体验,许多公司开放了接又,允许第三方应用使用用户数据进 行定制化服务。 

为了保护用户数据的安全和隐私,一直以来在计算机系统中有许多方法用于防止数据泄露。例如传统计算机系统中的ABAC、MAC、RBAC 等访问控制模型,互联网中的OCRID等身份管理认证机制,PKI公钥证书管理机制,DNS域名认证机制等。随着第三方应用的不断涌现,存储用户数据的数据服务器需要管理第三 方应用授权认证的系统。现在互联网上使用广泛的Oauth2.0框架,是一个第三方应用授权框架,能帮助用户向第三方应用供细粒度、动态可变的授权服务。Oauth2.0框架在Google、Facebook等平台得到广泛应用。

但是,已有的访问控制方法都存在一个共同的缺点,即依赖一个中心化的授权认证服务器负责权限管理。这使得授权认证完全由一个中心方控制。具体操作和管理细节对用户不透明,因而也不可信赖。一方面中心化的权限管理给攻击者供了攻击目标,一旦该中心存储的用户信息泄露,攻击者可以伪装用户访问数据库。另一方面,如果管理方内部出现问题,那么用户隐私将会受到严重威胁。区块链技术的出现让去中心化金融系统成为了现实。基于区块链技术的密码货币系统采用p2p网络以及共识算法,能让网络中的节点共同参与到交易的确认、区块的认证、以及区块链的存储和更新维护中来。使得用户可以不依赖传统的中心化金融机构如银行等,在一个去中心化系统中进行交易,保障了用户隐私和安全。 

\section{访问控制系统}

访问控制是计算机领域一个经典的问题,简单来说,计算机系统中的访问控制可以定义为防止对计算机中资源未授权的使用,包括未授权的用户对资源进行使用,或者对资源进行未授权的使用行为。访问控制系统用于管理数据的访问和操作权限,使得只有特定的用户才能进行特定的行为。传统的单机文件系统中通过对特定用户和用户组设置特定权限的方式进行文件权限的管理,而在分布式网络中,需要对不同的联网节点进行权限管理,到了互联网时代,网上应用的增加以及用户的多样化,对访问控制系统在安全性、灵活性等方面提出了更多的要求。

对于访问控制系统而言,其背后的核心访问控制模型决定了资源和用户之间连接关系的管理方式。这一领域传统的访问控制模型主要有自主访问控制(Discretionary Access Control,DAC),强制访问控制(Mandatory Access Control,MAC),基于角色的访问控制(Role-Based Access Control,RBAC),基于属性的访问控制(Attribute-Based Access Control,ABAC),不同的模型都用于解决资源和用户之间连接关系的问题。其中DAC模型指对每个资源和每个用户之间建立连接,指定不同的权限关系,主要应用于传统的单机文件系统。缺陷在于随着资源和用户数量的上升,访问控制列表的复杂度会急剧增加。MAC模型对每个资源和每个用户设置等级,每个用户可以访问等级较低的资源,这一模型主要用于军队等资源等级明确的场景。RBAC模型对不同的用户设置到各类角色,每一类角色拥有相同的权限,这一模型提升对一类用户修改权限以及对单一用户修改一类权限的效率。随着新的更灵活的权限范围不断出现,RBAC模型需要定义更多的角色。为了解决这一问题,ABAC模型将权限进一步细粒度化,对资源和角色都设定了一系列属性,通过属性访问控制列表对不同属性之间的关系进行管理。

在互联网时代,网络中各平台分别管理用户的不同数据,为了提升易用性,方便第三方应用访问用户数据,OAuth框架被各大平台广泛应用。该框架中,主要有客户端,数据所有者,数据服务器和授权服务器四种角色,其中数据所有者将数据存储在数据服务器,给第三方客户端进行授权,客户端通过授权服务器换取凭证,用于访问数据服务器的数据。

\section{区块链技术}

互联网技术保障了网络中节点间的可靠信息传输,但是无法保证传输信息的可信性,因而需要依赖中心化的权威节点保障节点间的信任。区块链技术旨在不可信的开放网络中,维护一个安全可信、不可篡改的公共账本,并以此做为信任基础构建电子交易、访问控制等应用系统。

根据新节点的加入是否需要授权认证,区块链系统可以分为许可链和非许可链两大类.非许可链通常也称为公有链,不限制节点的加入或退出,任何节点可以访问链上数据、发布交易、以及参与链上数据的记录,甚至可以尝试发布不合法消息,攻击网络中的其他节点。许可链指区块链网络中节点的加入网络、记录账本等操作需要经过特定的授权许可与认证。许可链系统又可以根据系统参与方的数量分为联盟链与私有链,其中联盟链由多方组织加入同一区块链网络中,共同维护区块链账本,记录并执行链上合约,多方组织可以通过一致的账本建立起联盟成员之间的信任。而私有链通常由一个参与方负责创建和维护,主要用于记录和管理内部核心数据,增强数据的安全性及可追溯性。

\section{隐私保护技术}

在区块链系统的实际使用中,为了保证区块链上记录数据的可溯源、可验证等特性,所有数据都必须公开给区块链网络中的所有节点。这一特性导致恶意攻击者可以直接获取区块链账本中记录的数据,通过分析区块链账本中记录的交易数据,发掘其中规律, 将用户的不同地址、交易数据关联,并进一步对应到用户的现实身份。 
	
近年来,许多研究者开始关注区块链系统中的隐私问题,该领域中相应的隐私保护技术也不断出现。本文通过技术实现原理,将保护技术划分为地址混淆,信息隐藏和通道隔离,并对各类技术抽象出通用模型,然后介绍各类 隐私保护技术的实现及对比.其中地址混淆机制通过交易交换不同用户的资产,对同一用户不同地址间的关 联关系进行混淆,从而破坏地址聚类的假设前提;信息隐藏机制通过零知识证明、同态加密等密码学技术加密 区块链账本中记录的隐私信息,同时保持账本正确性的可验证;通道隔离机制在区块链网络中设置访问权限, 将需要权限访问的数据保护在特定通道中。

\section{论文主要工作和组织结构}

本文研究了访问控制系统以及互联网中广泛应用的OAuth框架,并利用区块链技术解决OAuth框架里中心化授权服务器带来的性能瓶颈、安全性等问题。在对OAuth框架的增强中,提出了改进的区块链事务数据结构,适应基于属性的访问控制模型。另一方面,本文分析了现有区块链中的隐私保护技术,并提出了在访问控制数据中的应用技术。

本文第二章介绍了访问控制系统的传统模型以及OAuth2.0框架。第三章主要介绍了区块链领域的密码学及共识算法等基础知识。第四章提出了基于区块链技术的ABAC访问控制系统,详细描述了系统框架以及协议流程,并介绍了实现的原型系统以及测试数据结果。第五章分析了现有的各类隐私保护技术及其机制。第六章介绍了零知识证明技术与前文提出的访问控制系统结合,保障授权数据的隐私性。第七章对全文工作进行了总结并对未来工作进行了展望。
