% !TeX root = 。。/main。tex

\chapter{引言}
\label{cha:intro}

\section{背景意义}

互联网正逐步将现实社会变为数字化社会,越来越多的商业和社会活动从线下转移到线上,例如电子商务,电子办公等。同时,智能设备以及相关应用的种类也在逐渐增多,从电脑到手机、智能手环、智能电视、智能音箱等多种联网设备。随着用户使用各大平台的应用,大量个人数据被收集并且上传到互联网,这些数据存储在各大平台的服务器中。为了进一步丰富用户的使用体验,许多平台开放了数据接口,允许第三方应用使用用户数据进行定制化服务。

为了保护用户数据的安全和隐私,一直以来在计算机系统中有许多机制用于防止数据泄露。随着第三方应用的不断涌现,存储用户数据的平台需要高效、灵活、动听的授权服务系统。传统的授权服务系统在灵活性、扩展性等方面存在一定局限。目前,互联网上使用广泛的Oauth 2.0框架,能帮助用户向第三方应用供细粒度、动态可变的授权服务。该框架在Google、Facebook等平台得到广泛应用。已有的授权服务框架都存在一个共同的缺点,即依赖一个中心化的授权认证服务器负责权限的管理和授权的认证,这使得授权认证完全由一个中心方控制。中心化的权限管理给攻击者供了攻击目标,一旦该中心存储的用户信息泄露,攻击者可以伪装用户访问数据库。因此,保障系统中授权服务的安全性十分重要。

区块链技术的出现让去中心化金融系统成为了现实,基于区块链技术的密码货币系统采用p2p网络以及共识算法,能让网络中的节点共同参与到交易的确认、区块的认证、以及区块链的存储和更新维护中,使得用户可以不依赖传统的中心化金融机构如银行等。在一个去中心化系统中进行交易,保障了用户数据的安全。但这也导致用户数据在网络中各节点间传播和存储,增大数据隐私泄漏的风险。

\section{国内外相关研究现状}

\subsection{访问控制与授权认证}

访问控制是计算机领域一个经典的问题,简单来说,计算机系统中的访问控制可以定义为防止对计算机中资源未授权的使用,包括未授权的用户对资源进行使用,或者对资源进行未授权的使用行为。访问控制系统用于管理数据的访问和操作权限,使得只有特定的用户才能进行特定的行为。传统的单机文件系统中通过对特定用户和用户组设置特定权限的方式进行文件权限的管理,而在分布式网络中,需要对不同的联网节点进行权限管理,到了互联网时代,网上应用的增加以及用户的多样化,对访问控制系统在安全性、灵活性等方面提出了更多的要求。

访问控制的核心问题在于如何连接资源和拥有对应权限的用户,访问控制模型决定了用户和资源之间连接关系的管理方式。这一领域传统的访问控制模型主要有自主访问控制(Discretionary Access Control,DAC)模型,强制访问控制(Mandatory Access Control,MAC)模型,基于角色的访问控制(Role-Based Access Control,RBAC)模型,以及基于属性的访问控制(Attribute-Based Access Control,ABAC)模型。

其中DAC模型指对每个资源和每个用户之间建立连接,指定不同的权限关系,主要应用于单机文件系统。优势在于权限的设置灵活自由,缺陷在于随着资源数量和用户数量的上升,访问控制列表的复杂度会急剧增加,难以扩展到复杂的系统。MAC模型对每个资源和每个用户分别设置等级,按照等级的高低对用户的访问权限进行限定,优势在于管理简单,缺陷在于灵活性不足,主要适用于军队这类等级明确的场景。RBAC模型对不同的用户设置到各类角色,每一类角色拥有相同的权限,这一模型提升对同一类用户修改权限以及对单一用户修改一类权限的效率,适用于公司等分工明确的场景。缺陷在于随着新的更灵活的权限范围不断出现,角色数量不断增长。ABAC模型将权限进一步细粒度化,对资源、用户、操作以及环境都设定了一系列属性,通过属性访问控制列表对不同属性之间的关系进行管理。

除了访问控制系统以外,授权服务系统也十分重要。在互联网时代,网络中各平台分别管理用户的不同数据,为了提升用户数据的易用性,方便第三方应用访问用户数据,各大平台需要授权服务系统允许用户自主发起授权。目前广泛使用的OAuth 2.0授权服务框架中,主要有客户端,资源所有者,资源服务器和授权服务器四种角色。其中资源所有者将数据存储在资源服务器,可以授权客户端进行访问,客户端在获取授权后向授权服务器换取凭证,用于访问资源服务器的特定资源。

\subsection{区块链技术及隐私保护}

互联网技术保障了网络中节点间可靠的数据传输,但无法保证传输数据的可信性,需要依赖中心化的权威节点保障节点间的信任。区块链技术旨在不可信的开放网络中,维护一个安全可信、不可篡改的公共账本,并以此为信任基础构建电子交易、访问控制等应用系统。

根据新节点的加入是否需要授权认证,区块链系统可以分为许可链和非许可链两大类。非许可链通常也称为公有链,不限制节点的加入或退出,任何节点可以访问链上数据、发布交易、以及参与链上数据的记录,甚至可以尝试发布不合法消息,攻击网络中的其他节点。许可链指区块链网络中节点的加入网络、记录账本等操作需要经过特定的授权许可与认证。许可链系统又可以根据系统参与方的数量分为联盟链与私有链,其中联盟链由多方组织加入同一区块链网络中,共同维护区块链账本,记录并执行链上合约,多方组织可以通过一致的账本建立起联盟成员之间的信任。而私有链通常由一个参与方负责创建和维护,主要用于记录和管理内部核心数据,增强数据的安全性及可追溯性。

在区块链系统的实际使用中,为了保证区块链上记录数据的可溯源、可验证等特性,所有数据都必须公开给区块链网络中的所有节点。这一特性导致恶意攻击者可以直接获取区块链账本中记录的数据,通过分析区块链账本中记录的交易数据,发掘其中规律,将用户的不同地址、交易数据关联,并进一步对应到用户的现实身份。 

近年来,许多研究者开始关注区块链系统中的隐私问题,该领域中相应的隐私保护技术也不断出现。主要通过地址混淆,信息隐藏和通道隔离三类机制保护用户隐私。其中地址混淆机制通过交易交换不同用户的资产,对同一用户不同地址间的关联关系进行混淆,从而破坏地址聚类的假设前提。信息隐藏机制通过零知识证明、同态加密等密码学技术加密区块链账本中记录的隐私信息,同时保持账本正确性的可验证。通道隔离机制在区块链网络中设置访问权限,将需要权限访问的数据保护在特定通道中。

\subsection{基于区块链技术的访问控制系统}

近年来,部分研究尝试使用区块链技术结合访问控制系统。Dorri 等人将区块链用于存储访问控制策略,同时利用区块链不可篡改的特性生成一个时间顺序的不可变的事务历史,Alansari将区块链用于存储 访问控制策略的同时还存储了用户属性,Di Francesco等人也是将区块链用于保存访问控制策略。Shafagh 等人就是将区块链用于存储访问权限,保证权限不被篡改。Ouaddah 利用令牌表 示访问权限,令牌传递时,将访问控制策略以锁定脚本的方式嵌入到交易中,用户通过解锁脚本证明其拥有令牌。
2015年,Zyskind等人提出了一种基于区块链的手机应用授权系统,将区块链系统转化为访问控制管理系统。并与分布式离线存储DHT相结合。

相关研究主要存在以下问题:部分研究使用区块链账本存储用户数据、访问控制策略、用户权限等信息,而现有区块链账本的容量不足以支撑。本文提出的B-OAuth系统只存储授权数据在区块链上,用户数据、访问控制策略等数据仍存储在资源服务器。另一方面,部分研究仅使用现有的区块链数据结构存储数据,与访问控制和授权服务的需求并不一致。本文设计了针对访问控制和授权服务需求的数据结构,提升了系统的可用性。

\section{论文主要工作和组织结构}

本文研究了访问控制系统以及授权服务框架,并利用区块链技术解决现有服务授权框架里中心化授权服务器带来的性能瓶颈、安全性等问题。提出了基于区块链的访问控制和授权服务系统B-OAuth。

提出了改进的区块链事务数据结构,适应基于属性的访问控制模型。在此基础上,本文分析了现有区块链中的隐私保护技术,并提出了在访问控制数据中的应用技术。

本文第二章介绍了区块链技术,访问控制模型以及OAuth 2.0框架。第三章主要介绍了区块链领域的密码学及共识算法等基础知识。第四章提出了基于区块链技术的ABAC访问控制系统,详细描述了系统框架以及协议流程,并介绍了实现的原型系统以及测试数据结果。第五章分析了现有的各类隐私保护技术及其机制。第六章对全文工作进行了总结并对未来工作进行了展望。
