% !TeX root = ../main.tex

% 中英文摘要和关键字

\begin{abstract}
  访问控制领域是计算机系统中重要的研究领域,直接关系数据的安全和隐私。访问控制模型的发展已经较为成熟,DAC、MAC、RBAC、ABAC等访问控制模型被普遍应用。随着互联网应用的发展,各平台需要授权服务框架向第三方应用提供授权服务。目前,第三方授权服务框架OAuth 2.0被广泛应用,该框架未限制使用的访问控制模型类型,可以与各类访问控制系统方便地结合。

  但OAuth 2.0及类似的授权服务框架都存在严重中心化的授权服务器,这成为系统的性能瓶颈和安全隐患。针对这一问题,本文提出基于区块链技术的第三方授权服务框架B-OAuth,采用区块链网络替代OAuth框架中的授权服务器,提升系统安全性。并完成了原型系统和性能测试,验证了该设计的可行性。

  为了保护用户数据的隐私性,本文分析了现有区块链上隐私保护的技术和实现,将区块链上隐私保护技术归纳总结为地址混淆、信息隐藏、通道隔离三大机制,并详细介绍了各类隐私保护机制的原理、特征以及不同的实现方式。在此基础上,引入了隐私保护技术zk-snark,解决隐私风险问题。

  本文主要贡献如下:

  \begin{enumerate}
    \item 针对OAuth框架中存在的中心化授权服务器,本文提出了基于区块链网络的替代方案B-OAuth,增强授权服务系统的安全性和鲁棒性。针对区块链账本的存储模式,并根据访问控制系统的特点进行改进,定义符合授权认证特点的链上数据结构与状态数据结构。并完成了原型系统和性能测试,验证了该设计的可行性。
    \item 对现有区块链上的隐私保护技术进行分析,本文分析了现有区块链上隐私保护的技术和实现,将区块链上隐私保护技术归纳总结为地址混淆、信息隐藏、通道隔离三大机制,并详细介绍了各类隐私保护机制的原理、特征以及不同的实现方式。针对B-OAuth系统中存在的隐私风险问题,本文引入zk-snark技术,提出了对用户授权数据进行隐私保护的方法。
  \end{enumerate}

  % 关键词用“英文逗号”分隔
  \thusetup{
    keywords = {区块链, 访问控制, OAuth框架, 基于属性的访问控制模型, 隐私保护},
  }
\end{abstract}

\begin{abstract*}
  Access control is an important research field in computer science, which is directly related to the security and privacy of data. The development of access control models has been relatively mature, DAC, MAC, RBAC, ABAC and other access control models are widely used. With the development of Internet applications, platforms need authorization service frameworks to provide authorization services to third-party applications. At present, OAuth framework is widely used, the OAuth framework does not limit the use of access control model types, can be easily combined with various access control systems. 

  However, OAuth and similar authorization service frameworks all have a serious centralized authorization server, which becomes a performance bottleneck and security hazard of the system. To solve this problem, this paper proposes a third-party authorization service framework b-oauth based on block chain technology, and adopts block chain network to replace the authorization server in OAuth framework to improve system security. The prototype system and performance test are completed to verify the feasibility of the design. But the distributed storage of blockchain networks raises the risk of data leakage. 

  In order to protect the privacy of user data, this paper analyzes the existing technology and implementation of privacy protection on the block chain, and summarizes the privacy protection technology on the block chain into three mechanisms: address confusion, information hiding and channel isolation, and introduces in detail the principles, characteristics and different implementation methods of various privacy protection mechanisms. Zk-snark, a privacy protection technology suitable for b-oauth framework, is introduced to solve the privacy risk. The main contributions of this paper are as follows:

  \begin{enumerate}
    \item Aiming at the centralized authorization server in OAuth framework, this paper proposes b-oauth, an alternative scheme based on block chain network, to enhance the security and robustness of the authorization service system. According to the storage mode of blockchain ledger and the characteristics of access control system, the on-chain data structure and state data structure conforming to the characteristics of authorization and authentication are defined. The prototype system and performance test are completed to verify the feasibility of the design.
    \item Privacy protection technology to the existing block chain analysis, this paper analyzes the existing privacy protection technology and implementation of chain block, the block and privacy protection technology in the chain of generalizations to address confusion, information hiding, channel segregation three mechanisms, and introduces in detail the principle of all kinds of privacy protection mechanism, characteristics, and different ways of implementation. In view of the privacy risks in the b-oauth system, this paper introduces the zk-snark technology and proposes a privacy protection method for the authorized data of users.
  \end{enumerate}

  \thusetup{
    keywords* = {Blockchain, Access Control, OAuth Framework, Attribute-based Access Control Model, Privacy Protection},
  }
\end{abstract*}
