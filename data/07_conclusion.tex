% !TeX root = ../main.tex

\chapter{总结与展望}

\section{全文总结}

访问控制领域是计算机系统中重要的研究领域,直接关系数据的安全和隐私。访问控制模型的发展已经较为成熟,DAC、MAC、RBAC、ABAC等访问控制模型被普遍应用。随着互联网应用的发展,为了提升访问控制的灵活性,OAuth 2.0框架广泛用于互联网中向第三方应用提供授权服务。现有第三方授权服务框架都存在中心化的授权服务器,成为系统的安全隐患。

区块链技术能保障存储数据的安全性,但由于性能和存储有限,难以在各领域实际应用。访问控制领域的授权数据刚好满足这两点特性,一方面授权数据需要较高的安全性保障。另一方面,授权数据相较于业务数据而言规模较小。基于这些特点,本文提出基于区块链技术的访问控制系统BACS,主要进行以下技术的改进和创新:

\begin{enumerate}
	\item 针对现有第三方授权服务框架中存在的安全隐患,本文提出了基于区块链的访问控制系统BACS。BACS系统采用联盟链网络替代中心化授权服务器,通过多节点共同管理用户授权数据,增强访问控制系统的安全性和鲁棒性。BACS系统采用基于属性的访问控制模型,对区块链的链上数据结构和状态数据结构进行改进,使区块链技术适用于访问控制系统。
	\item 本文分析了现有区块链上隐私保护的技术和实现,将区块链上隐私保护技术归纳总结为地址混淆、信息隐藏、通道隔离三类机制。并详细介绍了各类隐私保护机制的原理、特征以及不同的实现方式。在BACS系统的基础上,进一步设计了隐私保护机制,引入zk-SNARK技术保护BACS系统中授权数据的隐私性,保障用户隐私安全。
\end{enumerate}

\section{未来展望}

基于区块链的访问控制系统未来有广泛的应用场景。本文提出了部分技术的改进,后续的研究可以从以下几个方面进行:

\begin{enumerate}
	\item 本文提出了基于属性的访问控制模型和区块链技术的结合,可以研究现有其他各访问控制模型基于区块链的实现,增强区块链技术对现有访问控制系统的兼容性。
	\item 本文采用了改进的PBFT共识协议,该共识协议在性能、扩展性方面还存在不足,难以应用于大规模网络。可以研究性能更高的共识协议,增强系统扩展性。
	\item 本文提出了对授权和操作进行隐私保护的设计和功能实现方案,还未将该方法结合现有系统进行完整的实现,可以将整个系统的存储数据结构进行修改,满足隐私保护特性。
\end{enumerate}