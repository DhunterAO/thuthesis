% !TeX root = ../main.tex

\chapter{总结与展望}

\section{全文总结}

访问控制领域是计算机系统中重要的研究领域,直接关系数据的安全和隐私。访问控制模型的发展已经较为成熟,DAC、MAC、RBAC、ABAC等访问控制模型被普遍应用。随着互联网应用的发展,为了提升访问控制的灵活性,OAuth 2.0框架广泛用于互联网中向第三方应用提供授权服务。OAuth及类似框架都存在严重中心化的授权服务器,成为系统的性能瓶颈和安全隐患。

区块链技术能保障存储数据的安全性,但由于性能和存储有限,难以在各领域实际应用。访问控制领域的授权数据刚好满足这两点特性,一方面授权数据需要较高的安全性保障,另一方面,授权数据相较于业务数据而言规模较小。基于这些特点,本文提出基于区块链技术的授权服务系统B-OAuth,主要进行以下技术的改进和创新:

\begin{enumerate}
	\item 针对授权服务框架中存在的中心化授权服务器,提出了区块链网络替代方案,增强系统的安全性和鲁棒性。
	\item 针对区块链账本的存储模式,对基于属性的访问控制系统进行改进,定义符合区块链特点的授权数据结构。
	\item 针对区块链网络存在的用户隐私泄漏风险,引入zk-snark技术,提出了对授权和操作进行隐私保护的方法。
\end{enumerate}

\section{未来展望}

基于区块链的访问控制系统未来有广泛的应用场景,但相关研究还较少。本文提出了一些技术的改进,但仍在框架设计、技术细节等方面存在一些需要改进的空间,后续的研究可以从以下几个方面进行:

\begin{enumerate}
	\item 本文只提出了基于属性的访问控制模型和区块链技术的结合,还可以研究现有其他各访问控制模型基于区块链的实现,增强该系统对现有访问控制系统的兼容性。
	\item 本文采用了改进的PBFT共识协议,该共识协议在性能、扩展性方面还存在不足,难以应用于多节点网络,因此还可以研究性能更高的共识协议。
	\item 本文只是提出了对授权和操作进行隐私保护的设计和功能实现方案,还未将该方法结合现有系统进行完整的实现,后续可以将整个系统的存储数据结构进行修改,满足隐私保护特性。
\end{enumerate}