% !TeX root = ../main.tex

\begin{resume}

  \resumeitem{个人简历}

  1995 年 07 月 25 日出生于 四川 省 资阳 市。

  2013 年 9 月考入 北京航空航天 大学 计算机科学与技术 系 计算机科学与技术 专业,2017 年 7 月本科毕业并获得 工学 学士学位。

  2017 年 9 月 研究生统一招生考试 进入 清华 大学 计算机科学与技术 系攻读 硕士 学位至今。

  \researchitem{发表的学术论文} % 发表的和录用的合在一起

  % 1. 已经刊载的学术论文(本人是第一作者,或者导师为第一作者本人是第二作者)
  \begin{publications}
    \item Zhang A., Bai X. (2020) Decentralized Authorization and Authentication Based on Consortium Blockchain. In: Zheng Z., Dai HN., Tang M., Chen X. (eds) Blockchain and Trustworthy Systems. BlockSys 2019. Communications in Computer and Information Science, vol 1156. Springer, Singapore (EI 收录, 检索号:20200708163648.)
  \end{publications}

  % 2. 尚未刊载,但已经接到正式录用函的学术论文(本人为第一作者,或者
  %    导师为第一作者本人是第二作者)。
  \begin{publications}[before=\publicationskip,after=\publicationskip]
    \item 张 奥, 白晓颖. 区块链隐私保护研究与实践综述 (已被 软件学报 录用. TH-CPL B类期刊.)
  \end{publications}

  % 3. 其他学术论文。可列出除上述两种情况以外的其他学术论文,但必须是
  %    已经刊载或者收到正式录用函的论文。
  % \begin{publications}
  %   \item Wu X M, Yang Y, Cai J, et al. Measurements of ferroelectric MEMS
  %     microphones. Integrated Ferroelectrics, 2005, 69:417-429. (SCI 收录, 检索号
  %     :896KM)
  %   \item 贾泽, 杨轶, 陈兢, 等. 用于压电和电容微麦克风的体硅腐蚀相关研究. 压电与声
  %     光, 2006, 28(1):117-119. (EI 收录, 检索号:06129773469)
  %   \item 伍晓明, 杨轶, 张宁欣, 等. 基于MEMS技术的集成铁电硅微麦克风. 中国集成电路,
  %     2003, 53:59-61.
  % \end{publications}

  % \researchitem{研究成果} % 有就写,没有就删除
  % \begin{achievements}
  %   \item 任天令, 杨轶, 朱一平, 等. 硅基铁电微声学传感器畴极化区域控制和电极连接的
  %     方法: 中国, CN1602118A. (中国专利公开号)
  %   \item Ren T L, Yang Y, Zhu Y P, et al. Piezoelectric micro acoustic sensor
  %     based on ferroelectric materials: USA, No.11/215, 102. (美国发明专利申请号)
  % \end{achievements}

\end{resume}
