% !TeX root = ../main.tex

\chapter{总结与展望}

\section{全文总结}

访问控制领域是计算机系统中重要的研究领域,直接关系数据的安全和隐私。访问控制模型的发展已经较为成熟,DAC、MAC、RBAC、ABAC等访问控制模型被普遍应用。随着互联网应用的发展,为了提升访问控制的灵活性,OAuth框架广泛用于互联网中向第三方应用提供授权服务。OAuth及类似框架都存在严重中心化的授权服务器,成为系统的性能瓶颈和安全隐患。为了解决这一问题,本文提出采用区块链技术,针对访问控制系统的一些关键技术进行改进和创新,主要贡献如下:

\begin{enumerate}
	\item 针对OAuth框架中存在的中心化授权服务器,提出了区块链网络替代方案,增强系统的安全性和鲁棒性。
	\item 针对区块链账本的存储模式,对基于属性的访问控制系统进行改进,定义符合区块链特点的授权数据结构。
	\item 改进传统的PBFT共识协议,能应用于联盟链的共识过程。
\end{enumerate}

在研究基于区块链技术的访问控制系统的过程中,分布式系统在保障数据安全性的同时,带来了数据泄露的隐私风险。为了保护用户数据在区块链系统中存储和使用的隐私性,本文研究了区块链上隐私保护的技术和实现,将现有的隐私保护技术归纳总结为地址混淆、信息隐藏、通道隔离三大类,并详细介绍了各类隐私保护机制的原理、特征以及不同的实现方式。现有的各类隐私保护机制 及实现技术从不同方面保护区块链隐私,因而在实际考虑隐私保护的区块链系统中,通常综合多种技术达到更全面的隐私保护效果。

\section{未来展望}

基于区块链的访问控制系统未来有广泛的应用场景,但相关研究还较少。本文提出了一些技术的改进,但仍在框架设计、技术细节等方面存在一些需要改进的空间:

\begin{enumerate}
	\item 本文只提出了基于属性的访问控制模型和区块链技术的结合,还需要研究现有其他各访问控制模型基于区块链的实现,增强该系统对现有访问控制系统的兼容性。
	\item 本文采用了改进的PBFT共识协议,该共识协议在性能、扩展性方面还存在不足,难以应用于大型访问控制系统,因此还需要研究其他适用的共识协议。
	\item 本文只研究了现有的区块链上隐私保护技术,还需要改进现有技术使得能应用于访问控制系统中。
\end{enumerate}